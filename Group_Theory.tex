\documentclass{Math_Note}

\title{Group Theory}
\author{Buce-Ithon}
\newdateformat{mydate}{\twodigit{\THEDAY}{ }\shortmonthname[\THEMONTH], \THEYEAR}
\date{\today}

\begin{document}
% Title page
\maketitle

% Content
\newpage
\tableofcontents
\newpage

% Surface-figure
\begin{figure}[H]
    \centering
    \includegraphics[scale=0.24]{"./Figures/Elma.png"}
    \caption{Beautiful Algebra To My Elma}
\end{figure}

% Chap0-Introduction
\setcounter{section}{-1}
\newpage
\section{What is Group?-Introduction to Group}
\textcolor{softcyan}{Accoding to Chiniese Wikipeida:} \\
\marginpar{\textcolor{cyan}{core of Group}}
In mathematics, the \textbf{Group} is a special \textcolor{lightblue}{set} equiped a \textcolor{lightblue}{binary operation}, (the operation) 
which has \textcolor{lightblue}{associativity}, \textcolor{lightblue}{identity} and \textcolor{lightblue}{inverse element}. \\
\newline
Since lots of mathematical structure are all groups (for instance: the integer system equiped with addition forms a Group), therefore, 
common results can be succinctly summarized from different mathematical structures, which makes groups a core concept in contemporary mathematics. \\
\newline
\textcolor{softcyan}{The unilateral (left) definition of Group (weak definition) is below:} \\
\marginpar{\textcolor{cyan}{simplest definition of the Group}}
If a given \textcolor{lightblue}{set} $G$, and it's equiped \textcolor{lightblue}{binary operation} $\circ$: $G\times G\rightarrow G$ satisfying 
(the result of operation $\circ\left(a,b\right)$ is simplified as $a\circ b$): \\
%\begin{table}[H]
%    \begin{center}
%        %\renewcommand\arraystretch{1.2}
%        \begin{tabular}{|c|l|l|} %option >{\columncolor[HTML]{F8F4FF}} adding before 'c,l' sets column color
%            \hline
%            \textcolor{lightblue}{Associativity} & \multicolumn{2}{l|}{$\forall g_{1}, g_{2},g_{3}\in G$, we have $\left(g_{1}\circ g_{2}\right)\circ g_{3}=g_{1}\circ\left(g_{2}\circ g_{3}\right)$} \\ \hline
%            \multirow{2}{*}{\makecell{(L)\textcolor{lightblue}{Identity}\\ \& (L)\textcolor{lightblue}{Inverse}}} & \multirow{2}{*}{$\exists e\in G, \forall g\in G$, we have} & $e\circ g=g$ \\ \cline{3-3}
%             & & and exists $\gamma\in G$, s.t. $\gamma\circ g=e$ \\ \hline
%        \end{tabular}
%    \end{center}
%\end{table}
%%\textcolor{wikitable1}{This is wikitable1 colored texts.} %wikitable1: \definecolor{wikitable1}{HTML}{F8F4FF}
%%\textcolor{wikitable2}{This is wikitable2 colored texts.} %wikitable2: \definecolor{wikitable2}{HTML}{D0E2FF}
\begin{table}[H]
    \begin{center}
        %\renewcommand\arraystretch{1.2}
        \begin{tabular}{|>{\columncolor[HTML]{F8F4FF}}c|>{\columncolor[HTML]{F8F4FF}}l|>{\columncolor[HTML]{F8F4FF}}l|} %option >{\columncolor[HTML]{F8F4FF}} adding before 'c,l' sets column color
            \hline
            \rowcolor{wikitable1} \textcolor{lightblue}{Associativity} & \multicolumn{2}{l|}{$\forall g_{1}, g_{2},g_{3}\in G$, we have $\left(g_{1}\circ g_{2}\right)\circ g_{3}=g_{1}\circ\left(g_{2}\circ g_{3}\right)$} \\ \hline
             & & $e\circ g=g$ \\ \cline{3-3}
            \multirow{-2}{*}{\makecell{(L)\textcolor{lightblue}{Identity}\\ \& (L)\textcolor{lightblue}{Inverse}}} & \multirow{-2}{*}{$\exists e\in G, \forall g\in G$, we have} & and exists $\gamma\in G$, s.t. $\gamma\circ g=e$ \\ \hline
        \end{tabular}
    \end{center}
\end{table}
%\textcolor{wikitable1}{This is wikitable1 colored texts.} %wikitable1: \definecolor{wikitable1}{HTML}{F8F4FF}
%\textcolor{wikitable3}{This is wikitable3 colored texts.} %wikitable3: \definecolor{wikitable3}{HTML}{D0E2FF}
then we say $\left(G,\circ\right)$ is a \textbf{Group}. If $\circ$ is understood, $\left(G,\circ\right)$ is simplified as $G$. \\
\newline
The order of group operations is very important, i.e $a\cdot b=b\cdot a$(commutative) doesn't necessarily hold. \\
A group satisfies the commutative law is called \textcolor{lightblue}{Commutative Group} (or \textcolor{lightblue}{Abelian Group} named after Niels$\cdot$Abel), those doesn't satisfy 
is called \textcolor{lightblue}{Non-commutative Group} (\textcolor{lightblue}{Non-Abelian Group}). (For instance: The Dihedra Group (appearing next chapter) is a Non-Abelian Group.) \\
\newline
\textcolor{softcyan}{Equivalance definition of Group: unilateral (right) definition of Group (strong definition) is below:}
\marginpar{\textcolor{cyan}{equivalance and full definition}}
The above definition part of identity and inverse can also be changed to: 
\begin{table}[H]
    \begin{center}
        \begin{tabular}{|c|l|l|}
            \hline
            \rowcolor{wikitable2} & & $g\circ e=g$ \\ \cline{3-3}
            \rowcolor{wikitable2} \multirow{-2}{*}{\makecell{\textcolor{lightblue}{Identity}(R)\\ \& \textcolor{lightblue}{Inverse}(R)}} & \multirow{-2}{*}{$\exists e\in G, \forall g\in G$, we have} & and exists $\gamma\in G$, s.t. $g\circ\gamma=e$ \\ \hline
        \end{tabular}
    \end{center}
\end{table}
Because no matter the original definition of lavender or the alternative definition of light-yellow, combined with the associativity, it's equivalent to the following definition: 
\begin{table}[H]
    \begin{center}
        \begin{tabular}{|c|l|l|}
            \hline
            \rowcolor{wikitable3} & & $e\circ g=g\circ e=g$ \\ \cline{3-3}
            \rowcolor{wikitable3} \multirow{-2}{*}{\makecell{\textcolor{lightblue}{Identity}\\ \& \textcolor{lightblue}{Inverse}}} & \multirow{-2}{*}{\makecell{$\exists e\in G, \forall g\in G$, then\\ we have}} & and exists $\gamma\in G$, s.t. $\gamma\circ g=g\circ\gamma=e$ \\ \hline
        \end{tabular}
    \end{center}
\end{table}
\textcolor{softcyan}{Ohh! Please remenber the definitions above, because this is the most concise definition of Group. And it will be very helpful for us to understand Group Theory!}

% Definition and examples
\newpage
\section{Definition of Group and some Examples}
\begin{df}{(Group)}
    If a set $G$ of some elements (finite or infinite) equiped with a binary operation $\circ$ (or $\cdot$, or even can be omitted), satisfies: 
    \begin{enumerate}
        \item $\forall a,b\in G$, $a\circ b\in G$, \textcolor{lightblue}{closure}
        \item $\forall a,b,c\in G$, $\left(a\circ b\right)\circ c=a\circ \left(b\circ c\right)$, \textcolor{lightblue}{associativity}
        \item $\exists e\in G$, $\forall g\in G$, $e\circ g=g\circ e=g$, \textcolor{lightblue}{identity (or neutral)}
        \item $\forall g\in G$, $\exists g'\in G$, $g'\circ g=g\circ g'=e$, \textcolor{lightblue}{inverse} \\
              what's more, we denote $g'=g^{-1}$. 
    \end{enumerate}
    Then, we denote $\left(G,\circ\right)$ is a \textbf{Group}.
\end{df}
\begin{ex*}
    \ % space taking place
    \begin{enumerate}
        \item $\left(\mathbb{Z},+\right)$ is a group, identity $e=0$, for some $g\in G$, its inverse $g^{-1}=-g$
        \item $\left(Q,+\right)$ is a group, similiar with $\left(\mathbb{Z},+\right)$ \\
              Special, $\left(Q^{\times},\times\right)$ is a group, with $Q^{\times}=Q/\{0\}$ and identity $e=1$
        \item $\left(\mathbb{R},+\right)$, $\left(\mathbb{R}^{\times},\times\right)$ with $\mathbb{R}^{\times}=\mathbb{R}/\{0\}$ 
              both are groups
        \item $\mathbb{Z}_{n}=\{m\vert m=x\ mod\ n,\ \forall x\in\mathbb{Z}\}=\{0,1,\cdots,n-1\}$, $\left(\mathbb{Z}_{n},+\right)$ 
              is a cyclic group 
              \marginpar{\textcolor{cyan}{Cyclic group: a group can be generated by at least one element of the group.}}
              \marginpar{\textcolor{skycyan}{Generate: Applying the binary operation of this group to the element, that is, 
              all integer powers of the element.}}
        \item Dihedra group: a group of symetries of regular polygon, with elements of rotations ($r_{i}$) and reflections ($s_{i}$), 
              often denoted by $D_{2n}$ (or $\left(D_{2n},\circ\right)$) or $D_{n}$ for $n$-gon polygon with $2n$ elements.
    \end{enumerate}
\end{ex*}

\end{document}
